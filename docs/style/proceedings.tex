\documentclass{sigchi}

% Arabic page numbers for submission.  Remove this line to eliminate
% page numbers for the camera ready copy
% \pagenumbering{arabic}

% Load basic packages
\usepackage{balance}       % to better equalize the last page
\usepackage{graphics}      % for EPS, load graphicx instead 
\usepackage[T1]{fontenc}   % for umlauts and other diaeresis
\usepackage{txfonts}
\usepackage{mathptmx}
\usepackage[pdflang={en-US},pdftex]{hyperref}
\usepackage{color}
\usepackage{booktabs}
\usepackage{textcomp}

% Some optional stuff you might like/need.
\usepackage{microtype}        % Improved Tracking and Kerning
% \usepackage[all]{hypcap}    % Fixes bug in hyperref caption linking
\usepackage{ccicons}          % Cite your images correctly!
% \usepackage[utf8]{inputenc} % for a UTF8 editor only

% If you want to use todo notes, marginpars etc. during creation of
% your draft document, you have to enable the "chi_draft" option for
% the document class. To do this, change the very first line to:
% "\documentclass[chi_draft]{sigchi}". You can then place todo notes
% by using the "\todo{...}"  command. Make sure to disable the draft
% option again before submitting your final document.
\usepackage{todonotes}

% Paper metadata (use plain text, for PDF inclusion and later
% re-using, if desired).  Use \emtpyauthor when submitting for review
% so you remain anonymous.
\def\plaintitle{SIGCHI Conference Proceedings Format}
\def\plainauthor{First Author, Second Author, Third Author,
  Fourth Author, Fifth Author, Sixth Author}
\def\emptyauthor{}
\def\plainkeywords{Authors' choice; of terms; separated; by
  semicolons; include commas, within terms only; this section is required.}
\def\plaingeneralterms{Documentation, Standardization}

% llt: Define a global style for URLs, rather that the default one
\makeatletter
\def\url@leostyle{%
  \@ifundefined{selectfont}{
    \def\UrlFont{\sf}
  }{
    \def\UrlFont{\small\bf\ttfamily}
  }}
\makeatother
\urlstyle{leo}

% To make various LaTeX processors do the right thing with page size.
\def\pprw{8.5in}
\def\pprh{11in}
\special{papersize=\pprw,\pprh}
\setlength{\paperwidth}{\pprw}
\setlength{\paperheight}{\pprh}
\setlength{\pdfpagewidth}{\pprw}
\setlength{\pdfpageheight}{\pprh}

% Make sure hyperref comes last of your loaded packages, to give it a
% fighting chance of not being over-written, since its job is to
% redefine many LaTeX commands.
\definecolor{linkColor}{RGB}{6,125,233}
\hypersetup{%
  pdftitle={\plaintitle},
% Use \plainauthor for final version.
%  pdfauthor={\plainauthor},
  pdfauthor={\emptyauthor},
  pdfkeywords={\plainkeywords},
  pdfdisplaydoctitle=true, % For Accessibility
  bookmarksnumbered,
  pdfstartview={FitH},
  colorlinks,
  citecolor=black,
  filecolor=black,
  linkcolor=black,
  urlcolor=linkColor,
  breaklinks=true,
  hypertexnames=false
}

% create a shortcut to typeset table headings
% \newcommand\tabhead[1]{\small\textbf{#1}}

% End of preamble. Here it comes the document.
\begin{document}

\title{\plaintitle}

\numberofauthors{2}
\author{%
	\alignauthor{Love Arreborn\\
		\email{lovar063@student.liu.se}}\\
	\alignauthor{Nadim Lakrouz\\
		\email{nadla777@student.liu.se}}\\
}

\maketitle

\section{1 INTRODUCTION}\label{introduction}

In cryptography, there are many applications for randomly generated
numbers. However, the process of producing these random numbers tends to
be pseudo-random, e.g.~utilizing the current states of various modules
{[}@randomness{]}. These numbers do not generate true randomness, and in
order to heighten security other methods of generating these are
required. Current random number generators (\emph{RNG}) are usually
implemented in code, using certain states of the host machine as a
starting point before running a predetermined algorithm
{[}@randomness{]}. This pseudo-random number generation (\emph{PRNG})
comes with the drawback that the result is always deterministic,
provided that the initial state is known.

True random numbers, then, cannot be produced solely through code. These
systems require some input that is neither replicable nor reproducible.
One proposed solution for this is quantum random number generation
(\emph{QRNG}) {[}@QRNG{]}. By reading quantum fluctuations from any
given source, for instance an optical signal, the inherent natural
unpredictability of said source can be harnessed in order to produce a
random number from a state that is nigh impossible to reproduce
accurately.

In this project, we will be writing firmware for one such solution,
which reads quantum variations from an optical signal. Further details
about how this signal is produced will be introduced in section 2 and
builds on the work of Clason {[}@Clason2023{]}. This optical signal will
be converted to a stream of random, raw bits via an Analog to Digital
Converter (\emph{ADC}). In turn, these random bits will be processed via
Toeplitz-hashing {[}@toeplitz{]} in order to process these bits into
random numbers. Some processing has to be done on the microcontrollers
themselves in order to ensure that the data is workable, and
Toeplitz-hashing is a tried and tested method to accomplish this. These
random numbers will then be output from the microcontroller to the host
computer via USB. This thesis will aim to answer one key research
question: How can sampled vacuum fluctuations be processed efficiently
in order to output QRNG?

In producing this firmware, several key considerations have to be made
in order for this system to be usable in a production environment. The
vision for the end product is a simple USB-stick that can be connected
to a host, and produce true random numbers from ambient quantum
fluctuations. While this system could very well be implemented on
physically larger hardware, thus avoiding the constraints that limited
hardware introduces, Clason proposes a simpler and cheaper way to
achieve QRNG {[}@Clason2023{]}. In keeping with this, utilizing
microcontrollers for processing the raw bits extracted further helps to
keep the costs low and the solution reasonably complex. Further details
regarding the microcontrollers used in our work will be outlined in
section 3. However, due to this portability constraint, our
implementation needs to work quickly and efficiently on resource
constrained hardware. As such, our main question is broken down into two
concrete research areas:

\textbf{Research area 1 (\emph{RA1})}: How can Toeplitz-hashing be
implemented as effectively as possible on resource constrained hardware
in order to process raw bits into a workable random number?

Toeplitz-hashing been optimized quite well, and previous research can be
utilized for this. However, there are still considerations when
implementing the firmware for the microcontroller in order to optimize
the code. Our goal is to attempt several implementations in order to
find the most optimal implementation with the least amount of
CPU-cycles.

\textbf{Research area 2 (\emph{RA2})}: How can we ensure that the output
of random numbers is not limited by our firmware, but rather only
limited by the USB transfer speed of the microcontroller or the ADC?

There will unequivocally be a bottleneck for the processing speed. For
instance, the speed at which the ADC can process the optical signal into
raw bits as well as the speed that the USB output can transfer processed
random number to the host computer will be limiting factors. Further
details on the limitations of the ADC will be outlined in section 3. The
slowest of these bottlenecks will inevitably be the limiting factor for
any implementation. Our research aims to ensure that our implementation
of Toeplitz-hashing does not become the limiting factor, but rather
processing data fast enough to match or exceed the speed of the
hardware.

\subsection{2 RELATED WORKS}\label{related-works}

This work is a practical continuation of the work of Clason
{[}@Clason2023{]}. In this work, the author aimed to study quantum shot
noise originating from photodiodes, and in so doing built a device which
read from an optical source, outputting analog voltage from the data
``seen'' by the diodes. A prototype was constructed, in which an LED is
read by a photodiode soldered millimeters apart.

The idea of an optical QRNG (\emph{OQRNG}) is not a novel one. The basis
of the theory is that intrinsically random properties of a quantum
process. Stefanov et. al.~proposes using the random choice of a photon
between two output signals to generate a random stream of bits, however
the theory behind it can be applied to other quantum processes as well
{[}@StefanovOptical{]}. This particular theorem has been implemented by
Wayne et. al.~to create a quantum number generator {[}@Wayne{]}. While
this article proves the efficacy of OQRNG, it utilizes a slightly
different method.

Our work revolves around the measurement of shot noise of vacuum states
rather than measuring arrival times of photons. Essentially, this is
another quantum process with the same inherently random properties as
described by Stefanov et. al., but instead using shot noise. As
described by Niemczuk {[}@shotnoise{]}, shot noise is minor fluctuations
in an elecritcal current, which is inherently random. Reading this
property, then, gives us an intrisically random source from which to
generate a random output, which in turn can be processed into a random
number.

Implementations of this theory exist, however with significant
drawbacks. Shen et. al.~presents an implementation using a fairly
complex setup, in which a continuous-wave fiber laser is the optical
source {[}@contender1{]}. They conclude that sampling the shot noise is,
indeed, suitable for OQRNG. However, the implementation requires
expensive and complex hardware, and the sheer size of the system
prohibits it from being portable and easily reproducible in small-scale
tests.

A more recent implementation of OQRNG in a smaller scale has been
presented by Singh et. al. {[}@singh{]}. This particular implementation
uses a bespoke circuit board where all components are present on a
single board -- e.g., this experimental setup contains an integrated
ADC, post-processor, entropy controller and entropy generator. While
this article cements the viability of OQRNG using shot noise (despite
the article not being confirmed as peer reviewed), the bespoke nature of
the circuit board makes this experiment difficult to reproduce. As our
thesis will use commercially available ADCs and microcontrollers, the
only bespoke component is the shot noise generator itself. Furthermore,
the Toeplitz-hashing is not run on the microcontroller itself in these
experiments -- instead, the hashing of these raw bits is done on the
receiving computer as this bespoke circuit board featured a relatively
weak processor.

In summary, previous research has proven that OQRNG can generate true
randomness, and more specifically, Shen et. al. {[}@contender1{]} and
Singh et. al. {[}@singh{]} both implement OQRNG through readings of shot
noise. However, there are limitations in both of these works. Either the
system that generates the shot noise is large and complex
{[}@contender1{]} or the system is built on bespoke hardware with
limitations in processing power which prevents a fully integrated system
{[}@singh{]}. Our work aims to bridge this gap by using commercially
available hardware (other than the bespoke shot noise generator
{[}@Clason2023{]}) and focuses on implementing Toeplitz-hashing directly
on the microcontroller.

\pagebreak
% BALANCE COLUMNS
\balance{}

% REFERENCES FORMAT
% References must be the same font size as other body text.
\bibliographystyle{SIGCHI-Reference-Format}
\bibliography{sample}

\end{document}

%%% Local Variables:
%%% mode: latex
%%% TeX-master: t
%%% End:
