\documentclass[noamsthm, english, xcolor=x11names, xcolor=svgnames, xcolor=dvipsnames]{beamer}
\usepackage[swedish]{babel}
\usepackage{etoolbox} % set
\usepackage{parskip}
\usepackage{graphicx,txfonts}
\usepackage[T1]{fontenc}
\usepackage{enumitem}
\usepackage{mathabx}
\usepackage{alltt}
\usetheme[
  dark,%
  %% Standard colors {blue, green, turquoise} for title, outline and end pages
  titlecolor=blue,%
  outlinecolor=green,%
  endcolor=turquoise,%
  %% Complementary colors {orange, purple, yellow, gray} for some details
  complementary=orange,%
  %% All colors {blue, green, turquoise, orange, purple, yellow, gray} for blocks
  blockcolor=blue,%
  %% Font schemes {latex, standard, sansserif, calibri, liu} to use
  font=calibri,%
  %license=ccby,%
  %% Select color themes {highcontrast, superhighcontrast} with high contrast or add handout option in documentclass
  %highcontrast,%
  %superhighcontrast,%
  %% Remove outline before each section and end page with {nooutline, noendpage}
  outline,%
  %nooutline, %
  %noendpage,%
  %% Add navigation symbols with {navigation}
  %navigation,%
  %% Add total number of frames to frame count with {totalframes}
  totalframes,%
  % Remove parts of header with {noheadertitle, noheaderauthor, noheaderdate, noheadernumber, minimalheader, noheader}
  noheadertitle,%
  noheaderauthor,% 
  noheaderdate,%
  %noheadernumber,%
  %% Show institute in title frame
  showinstitute,%
  %% Show outline in two columns
  %outlinecolumns,
  %footerimage,
]{LiU}

% make LiU-logo black
\makeatletter
\def\LiU@logo@color@footline{black}
\makeatother

\usepackage{listings}
\usepackage{tikz}
\usepackage{../tikz-uml}
\usepackage[x11names]{xcolor}
\usepackage[forceshell,pgf,debug,outputdir={graphs/}]{dot2texi}

\setitemize{label=\usebeamerfont*{itemize item}%
  \usebeamercolor[fg]{itemize item}
  \usebeamertemplate{itemize item}}
\setenumerate[1]{%
  label=\protect\usebeamerfont{enumerate item}%
        \protect\usebeamercolor[fg]{enumerate item}%
        \insertenumlabel.}
\setenumerate[2]{%
  label=\protect\usebeamerfont{enumerate subitem}%
        \protect\usebeamercolor[fg]{enumerate subitem}%
        \insertsubenumlabel.}
%% \setenumerate[2]{%
%%   label=\protect\usebeamerfont{enumerate subitem}%
%%         \protect\usebeamercolor[fg]{enumerate subitem}%
%%         (\alph{enumii})}

%% Add a text on end page, leave blank to add author
\finaltext{}

\usetikzlibrary{shapes,arrows}
\usetikzlibrary{calc, tikzmark, fit}
\usetikzlibrary{positioning, decorations.markings}
\usetikzlibrary{patterns,snakes}

\setmonofont{Liberation Mono}
\lstset{basicstyle=\scriptsize\fontspec{LiberationMono}}
\date{}

\definecolor{LiUBlue}{RGB}{0,185,231}
\definecolor{LiUTurquoise}{RGB}{23,199,210}
\definecolor{LiUGreen}{RGB}{0,207,181}
\definecolor{LiUOrange}{RGB}{255,100,66}
\definecolor{LiUPurple}{RGB}{137,129,211}
\definecolor{LiUYellow}{RGB}{253,239,93}
\definecolor{LiUGray}{RGB}{106,126,145}

\tikzumlset{font=\tiny}

\lstdefinelanguage[11]{C++}[ISO]{C++}{
  morekeywords={alignas,alignof,char16_t,char32_t,constexpr,override,%
    decltype,noexcept,nullptr,static_assert,thread_local,final, nullopt},%
}

%\documentclass[english]{liu}
\lstset{ %
  language=ruby, %
  commentstyle=\color{ForestGreen}, %
  keywordstyle=\color{blue}, %
  stringstyle=\color{BrickRed}, %
  captionpos=t, %
  frame=single, %
  tabsize=2, %
  rulecolor=\color{black}, %
  showstringspaces=false, %
  backgroundcolor=\color{WhiteSmoke},%
  extendedchars=true, %
  escapeinside={@(}{)@},          % if you want to add LaTeX within your code
  basicstyle=\ttfamily\footnotesize 
}


\lstdefinestyle{run}{%
  frame=single,%
  backgroundcolor=\color{Ivory1},%
  language=bash,%
  showstringspaces=false, %
  escapeinside={@(}{)@},          % if you want to add LaTeX within your code
}

\lstdefinelanguage{none}{
  identifierstyle=
}

\lstdefinestyle{terminal}{%
  frame=single,%
  backgroundcolor=\color{Ivory1},%
  language=none,%
  showstringspaces=false, %
  escapeinside={@(}{)@},          % if you want to add LaTeX within your code
  basicstyle=\ttfamily\tiny
}

\lstdefinestyle{noterminal}{%
  style=terminal,%
  frame=none,
  backgroundcolor=\color{white},%
}

\lstdefinestyle{small}{%
  basicstyle=\ttfamily\tiny
}

\lstdefinestyle{medium}{%
  basicstyle=\fontsize{7}{9}\selectfont\ttfamily
}

\lstdefinestyle{smallrun}{%
  style=run,
  basicstyle=\ttfamily\tiny
}

\lstdefinestyle{basic}{
  style=small,
  backgroundcolor=\color{white},%
  frame=none,
}

\hypersetup{
	colorlinks=true,
        urlcolor=cyan,
        linkcolor=.
}

\setbeamercolor*{section in toc}{fg=black}
\setbeamercolor*{subsection in toc}{fg=black}
\setbeamercolor*{section in toc shaded}{fg=LiUgreen0!80!black}
\setbeamercolor*{subsection in toc shaded}{fg=LiUgreen0!80!black}
\setbeamertemplate{section in toc shaded}{\leavevmode\leftskip=2.5em{\hspace{-2.5em}\makebox[2em]{\inserttocsectionnumber}\inserttocsection\par}}
\setbeamertemplate{subsection in toc shaded}{\leavevmode\leftskip=2.9em{\hspace{-0.5em}\inserttocsubsection\par}}

% Use lstinline as item in description
\makeatletter
\newcommand*{\lstitem}[1][]{%
  \setbox0\hbox\bgroup
    \patchcmd{\lst@InlineM}{\@empty}{\@empty\egroup\item[\usebox0]\leavevmode\ignorespaces}{}{}%
    \lstinline[#1]%
}
\makeatother

\newcommand{\lst}[1]{\lstinline{#1}}
\newcommand{\codestyle}[1]{\footnotesize\fontspec{LiberationMono}\texttt{#1}}
\newcommand{\smallcodestyle}[1]{\tiny\fontspec{LiberationMono}\texttt{#1}}
\newfontface\extrafont{Liberation Serif}

\tikzset{fun/.style={inner sep=0, anchor=west, outer sep=0}}

\tikzset{var/.style={minimum height=0.4cm, minimum width=0.4cm, font=\scriptsize\ttfamily}}
\tikzset{var val/.style={var, rectangle, draw=black, font=\tiny\ttfamily}}
\tikzset{myptr/.style={>=latex, decoration={markings,mark=at position 1 with %
      {\arrow[scale=2]{>}}},postaction={decorate}}}
\tikzset{array element/.style={rectangle, thick, draw=black, minimum width=0.7cm, minimum height=0.7cm, font=\ttfamily}}

\tikzset{class/.style={rectangle, draw=black, anchor=center}}

\newcommand{\varbox}[3]
{
  % #1: Name of this variable
  % #2: Value of this variable
  % #3: position
  \node[var, anchor=east] (#1 label) at (#3) {\texttt{#1}};
  \node[var val, anchor=west, xshift=-\pgflinewidth] (#1 value) at (#1 label.east) {\texttt{#2}};
  \node[fit=(#1 label)(#1 value), inner sep=0] (#1) {};
}

\newcommand{\fullvarbox}[4]
{
  % #1: Name of this shape
  % #2: Name of the variable
  % #3: Value of this variable
  % #4: position
  \node[var, anchor=east] (#1 label) at (#4) {\texttt{#2}};
  \node[var val, anchor=west, xshift=-\pgflinewidth] (#1 value) at (#1 label.east) {\texttt{#3}};
  \node[fit=(#1 label)(#1 value), inner sep=0] (#1) {};
}

\newcommand{\extendedvarbox}[4]
{
  % #1: Name of this variable
  % #2: Value of this variable
  % #3: position
  % #4: data type
  \node[var, anchor=east] (#1 label) at (#3) {\texttt{#1}};
  \node[var val, anchor=west, xshift=-\pgflinewidth] (#1 value) at (#1 label.east) {\texttt{#2}};
  \node[var, anchor=north] (#1 type) at (#1 value.south) {\lstinline[style=small]{#4}};
  \node[fit=(#1 label)(#1 value)(#1 type), inner sep=0] (#1) {};
}

\newcommand{\containerbox}[5]
{
  % #1: name of containerbox
  % #2: left most varbox
  % #3: right most varbox
  % #4: top most varbox
  % $5: bottom most varbox

  \draw[thick] let 
  \p{left} = ($(#2.west) - (0.05cm, 0)$), 
  \p{right} = ($(#3.east) + (0.025cm, 0)$),
  \p{top} = ($(#4.north) + (0, 0.05cm)$),
  \p{bottom} = ($(#5.south) - (0, 0.05cm)$),
  \p{top left} = (\x{left}, \y{top}),
  \p{bottom right} = (\x{right}, \y{bottom}),
  \p{bottom left} = (\x{left}, \y{bottom}),
  \p{midpoint} = ($0.5*(\p{bottom left}) + 0.5*(\p{bottom right})$),
  \n{width} = {\x{right} - \x{left}},
  \n{height} = {\y{top} - \y{bottom}}
  in
  node[anchor=north west, minimum width=\n{width}, font=\scriptsize, inner sep=0.1cm] (#1 label) at (\p{bottom left}) {#1}
  node[anchor=north west, rectangle, draw=black, thick, minimum width=\n{width}, minimum height={\n{height}}] (#1 container) at (\p{top left}) {}
  node[fit=(#1 container)(#1 label), inner sep = 0] (#1) {};
}

\newcommand{\nullptr}[1]{
  % #1: which box to make null
  \draw ($(#1.north west) + (\pgflinewidth, -\pgflinewidth)$) -- ($(#1.south east) + (-\pgflinewidth, \pgflinewidth)$);
}

\newcommand{\unused}[1]{
  % #1: which box to make null
  \draw ($(#1.north west) + (0.1cm, -0.1cm)$) -- ($(#1.south east) + (-0.1cm, 0.1cm)$);
  \draw ($(#1.north east) + (-0.1cm, -0.1cm)$) -- ($(#1.south west) + (0.1cm, 0.1cm)$);
}

\tikzset{box/.style={minimum height=0.5cm, minimum width=0.5cm, rectangle, draw=black}}
\tikzset{ptrbox/.style={box, minimum width=0.2cm}}

\newcommand{\nodebox}[3]{
  % #1: name
  % #2: position
  % #3: value
  \node[box, anchor=center] (#1 value) at (#2) {\scriptsize\texttt{#3}};
  \node[ptrbox, anchor=west, xshift=-\pgflinewidth] (#1 next) at (#1 value.east) {};
  \node[inner sep=0, fit=(#1 value)(#1 next)] (#1) {};
}

\newcommand{\pointer}[2]{
  % #1: from
  % #2: to
  \draw[->, >=latex] (#1 next.center) -- (#2.west);
}


\newcommand{\binodebox}[3]{
  % #1: name
  % #2: position
  % #3: value
  \node[box, anchor=center] (#1 value) at (#2) {\scriptsize\texttt{#3}};
  \node[ptrbox, anchor=east, xshift=\pgflinewidth] (#1 prev) at (#1 value.west) {};
  \node[ptrbox, anchor=west, xshift=-\pgflinewidth] (#1 next) at (#1 value.east) {};
  \node[inner sep=0, fit=(#1 prev)(#1 value)(#1 next)] (#1) {};
}

\newcommand{\bipointer}[2]{
  % #1: from
  % #2: to
  \draw[->, >=latex, transform canvas={yshift=0.1cm}] (#1 next.center) -- (#2.west);
  \draw[<-, >=latex, transform canvas={yshift=-0.1cm}] (#1.east) -- (#2 prev.center);
}

\newcommand{\fullcontainerbox}[6]
{
  % #1: name of containerbox
  % #2: label of containerbox
  % #3: left most varbox
  % #4: right most varbox
  % #5: top most varbox
  % $6: bottom most varbox

  \draw[thick] let 
  \p{left} = ($(#3.west) - (0.05cm, 0)$), 
  \p{right} = ($(#4.east) + (0.025cm, 0)$),
  \p{top} = ($(#5.north) + (0, 0.05cm)$),
  \p{bottom} = ($(#6.south) - (0, 0.05cm)$),
  \p{top left} = (\x{left}, \y{top}),
  \p{bottom right} = (\x{right}, \y{bottom}),
  \p{bottom left} = (\x{left}, \y{bottom}),
  \p{midpoint} = ($0.5*(\p{bottom left}) + 0.5*(\p{bottom right})$),
  \n{width} = {\x{right} - \x{left}},
  \n{height} = {\y{top} - \y{bottom}}
  in
  node[anchor=north west, minimum width=\n{width}, font=\scriptsize, inner sep=0.1cm] (#1 label) at (\p{bottom left}) {#2}
  node[anchor=north west, rectangle, draw=black, thick, minimum width=\n{width}, minimum height={\n{height}}] (#1 container) at (\p{top left}) {}
  node[fit=(#1 container)(#1 label), inner sep = 0] (#1) {};
}

\makeatletter
\newcommand{\createarray}[4]{%
  % #1: x
  % #2: y
  % #3: name
  % Every other argument is a new box and its content
  \edef\arrstart{#1}
  \edef\arrheight{#2}%
  \edef\arrname{#3}%
  \pgfmathtruncatemacro{\arrcounter}{0}%
  \node[array element] (\arrname \arrcounter) at ({\arrstart + 0.7*\arrcounter}, \arrheight) {#4};%
  \pgfmathtruncatemacro{\arrcounter}{\arrcounter + 1}%
  \arraynextarg%
}
\newcommand{\arraynextarg}{\@ifnextchar\bgroup{\arraygobblearg}{}}
\newcommand{\arraygobblearg}[1]{%
  \node[array element] (\arrname \arrcounter) at ({\arrstart + 0.7*\arrcounter}, \arrheight) {#1};%
  \pgfmathtruncatemacro{\arrcounter}{\arrcounter + 1}%
  \@ifnextchar\bgroup{\arraygobblearg}{}}
\makeatother

\makeatletter
\newcommand{\createverticalarray}[4]{%
  % #1: x
  % #2: y
  % #3: name
  % Every other argument is a new box and its context
  \edef\arrstart{#2}
  \edef\arrwidth{#1}%
  \edef\arrname{#3}%
  \pgfmathtruncatemacro{\arrcounter}{0}%
  \node[array element] (\arrname \arrcounter) at (\arrwidth, {\arrstart - 0.7*\arrcounter}) {#4};%
  \pgfmathtruncatemacro{\arrcounter}{\arrcounter + 1}%
  \verticalarraynextarg%
}
\newcommand{\verticalarraynextarg}{\@ifnextchar\bgroup{\verticalarraygobblearg}{}}
\newcommand{\verticalarraygobblearg}[1]{%
  \node[array element] (\arrname \arrcounter) at (\arrwidth, {\arrstart - 0.7*\arrcounter}) {#1};%
  \pgfmathtruncatemacro{\arrcounter}{\arrcounter + 1}%
  \@ifnextchar\bgroup{\verticalarraygobblearg}{}}
\makeatother


\newcommand{\underscore}[2]{
  \draw[draw=#2] ($(pic cs:#1beg) - (0.05cm, 0.05cm)$) -- ($(pic cs:#1end) + (0.05cm, -0.05cm)$);
  \draw let 
  \p{left} = ($(pic cs:#1beg) - (0.05cm, 0.05cm)$), 
  \p{right} = ($(pic cs:#1end) + (0.05cm, -0.05cm)$),
  \p{mid} = ($0.5*(\p{left}) + 0.5*(\p{right})$)
  in
  (\p{left}) -- (\p{right})
  coordinate (#1) at (\p{mid});
}

\newcommand{\highlight}[2]{\draw[draw=#2, thick] ($(pic cs:#1beg) - (0.05cm, 0.1cm)$) rectangle ($(pic cs:#1end) + (0.05cm, 0.1cm) + (0, 1ex)$);}

\newcommand{\smallhighlight}[2]{\draw[draw=#2, thick] ($(pic cs:#1beg) - (0.025cm, 0.05cm)$) rectangle ($(pic cs:#1end) + (0.025cm, 0.05cm) + (0, 0.7ex)$);}

\newcommand{\highlightbox}[2]{
  \coordinate (#1 bottomleft) at ($(pic cs:#1beg) + (0.1cm, 0)$);
  \coordinate (#1 topright) at ($(pic cs:#1end) + (-0.1cm, 1ex)$);
  \node[fit=(#1 bottomleft)(#1 topright), draw=#2, thick] (#1) {};
}

\newcommand{\pointto}[5]{\draw[->, >=stealth, thick, draw=red, red] (#1) to [#5] node[midway, #4, red]{#3}(#2);}

\newcommand{\crossout}[1]{\draw[draw=black, thick] ($(pic cs:#1beg) + (0, 1ex)$) -- ($(pic cs:#1end)$);\draw[draw=black, thick] ($(pic cs:#1end) + (0, 1ex)$) -- ($(pic cs:#1beg)$);}

\newcommand{\vtable}[4]
{
  \node[align=left, anchor=north west, font=\scriptsize\ttfamily, inner sep=0] (#1 vtable) at (#2) {
    \begin{tabular}{|l|}
      \hline
      \textbf{vtable for #1} \\
      \hline
      #3 \\
      \hline
    \end{tabular}
  };
}

\newcommand{\pointtocode}[3]
{
  \draw[->, thick,draw=#3] ($(pic cs:#1) + (0, 0.3ex)$) -- ($(pic cs:#2) + (0, 0.3ex)$);
}

\title{Efficient Exctraction of True Random Numbers from Quantum System on Resource Constrained Hardware}
\author{Love Arreborn and Nadim Lakrouz}
\institute{Institutionen för datavetenskap}
